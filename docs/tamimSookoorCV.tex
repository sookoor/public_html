%%%%%%%%%%%%%%%%%%%%%%%%%%%%%%%%%%%%%%%%%%%%%%%%%%%%%%%%%%%%%%%%%%%%%%%%
%%%%%%%%%%%%%%%%%%%%%% Simple LaTeX CV Template %%%%%%%%%%%%%%%%%%%%%%%%
%%%%%%%%%%%%%%%%%%%%%%%%%%%%%%%%%%%%%%%%%%%%%%%%%%%%%%%%%%%%%%%%%%%%%%%%

%%%%%%%%%%%%%%%%%%%%%%%%%%%%%%%%%%%%%%%%%%%%%%%%%%%%%%%%%%%%%%%%%%%%%%%%
%% NOTE: If you find that it says                                     %%
%%                                                                    %%
%%                           1 of ??                                  %%
%%                                                                    %%
%% at the bottom of your first page, this means that the AUX file     %%
%% was not available when you ran LaTeX on this source. Simply RERUN  %% 
%% LaTeX to get the ``??'' replaced with the number of the last page  %% 
%% of the document. The AUX file will be generated on the first run   %%
%% of LaTeX and used on the second run to fill in all of the          %%
%% references.                                                        %%
%%%%%%%%%%%%%%%%%%%%%%%%%%%%%%%%%%%%%%%%%%%%%%%%%%%%%%%%%%%%%%%%%%%%%%%%

%%%%%%%%%%%%%%%%%%%%%%%%%%%% Document Setup %%%%%%%%%%%%%%%%%%%%%%%%%%%%

% Don't like 10pt? Try 11pt or 12pt
%\documentclass[10pt]{article}
\documentclass[letterpaper]{article}

% This is a helpful package that puts math inside length specifications
\usepackage{calc}

% Layout: Puts the section titles on left side of page
\reversemarginpar

%
%         PAPER SIZE, PAGE NUMBER, AND DOCUMENT LAYOUT NOTES:
%
% The next \usepackage line changes the layout for CV style section
% headings as marginal notes. It also sets up the paper size as either
% letter or A4. By default, letter was used. If A4 paper is desired,
% comment out the letterpaper lines and uncomment the a4paper lines.
%
% As you can see, the margin widths and section title widths can be
% easily adjusted.
%
% ALSO: Notice that the includefoot option can be commented OUT in order
% to put the PAGE NUMBER *IN* the bottom margin. This will make the
% effective text area larger.
%
% IF YOU WISH TO REMOVE THE ``of LASTPAGE'' next to each page number,
% see the note about the +LP and -LP lines below. Comment out the +LP
% and uncomment the -LP.
%
% IF YOU WISH TO REMOVE PAGE NUMBERS, be sure that the includefoot line
% is uncommented and ALSO uncomment the \pagestyle{empty} a few lines
% below.
%

%% Use these lines for letter-sized paper
\usepackage[paper=letterpaper,
            %includefoot, % Uncomment to put page number above margin
            marginparwidth=1.2in,     % Length of section titles
            marginparsep=.05in,       % Space between titles and text
            margin=1in,               % 1 inch margins
            includemp]{geometry}

%% Use these lines for A4-sized paper
%\usepackage[paper=a4paper,
%            %includefoot, % Uncomment to put page number above margin
%            marginparwidth=30.5mm,    % Length of section titles
%            marginparsep=1.5mm,       % Space between titles and text
%            margin=25mm,              % 25mm margins
%            includemp]{geometry}

%% More layout: Get rid of indenting throughout entire document
\setlength{\parindent}{0em}

%% This gives us fun enumeration environments. compactenum will be nice.
\usepackage{paralist}

%% Reference the last page in the page number
%
% NOTE: comment the +LP line and uncomment the -LP line to have page
%       numbers without the ``of ##'' last page reference)
%
% NOTE: uncomment the \pagestyle{empty} line to get rid of all page
%       numbers (make sure includefoot is commented out above)
%
\usepackage{fancyhdr,lastpage}
\pagestyle{fancy}
%\pagestyle{empty}      % Uncomment this to get rid of page numbers
\fancyhf{}\renewcommand{\headrulewidth}{0pt}
\fancyfootoffset{\marginparsep+\marginparwidth}
\newlength{\footpageshift}
\setlength{\footpageshift}
          {0.5\textwidth+0.5\marginparsep+0.5\marginparwidth-2in}
\lfoot{\hspace{\footpageshift}%
       \parbox{4in}{\, \hfill %
                    \arabic{page} of \protect\pageref*{LastPage} % +LP
%                    \arabic{page}                               % -LP
                    \hfill \,}}

% Set your name here
\def\name{Tamim I. Sookoor}

% Finally, give us PDF bookmarks
\usepackage{color,hyperref}
\definecolor{darkblue}{rgb}{0.0,0.0,0.3}
\hypersetup{
  letterpaper,
  colorlinks = true,
  breaklinks,
  linkcolor=darkblue,
  urlcolor = black,
  anchorcolor=darkblue,
  citecolor=darkblue,
  pdfpagemode=node,
  pdftitle={\name: Curriculum Vitae},
  pdfauthor={\name},
  pdfsubject={Curriculum Vitae},
  pdfkeywords={computer science academia sensor networks
    macroprogramming routing cyber physical systems
    reliability latency bound university of virginia
    quality of service tamim imran sookoor kamin
    whitehouse tinyos nesc java linux unix matlab}
}

\geometry{textheight=8.5in, textwidth=6in}

% Customize page headers
\pagestyle{myheadings}
\markright{\name}
\thispagestyle{empty}

% Customize section headings
\usepackage{sectsty}
\subsectionfont{\rmfamily\mdseries\itshape\large}

%%%%%%%%%%%%%%%%%%%%%%%% End Document Setup %%%%%%%%%%%%%%%%%%%%%%%%%%%%


%%%%%%%%%%%%%%%%%%%%%%%%%%% Helper Commands %%%%%%%%%%%%%%%%%%%%%%%%%%%%

% The title (name) with a horizontal rule under it
%
% Usage: \makeheading{name}
%
% Place at top of document. It should be the first thing.
\newcommand{\makeheading}[1]%
        {\hspace*{-\marginparsep minus \marginparwidth}%
         \begin{minipage}[t]{\textwidth+\marginparwidth+\marginparsep}%
                {\large \bfseries #1}\\[-0.15\baselineskip]%
                 \rule{\columnwidth}{1pt}%
         \end{minipage}}

% The section headings
%
% Usage: \section{section name}
%
% Follow this section IMMEDIATELY with the first line of the section
% text. Do not put whitespace in between. That is, do this:
%
%       \section{My Information}
%       Here is my information.
%
% and NOT this:
%
%       \section{My Information}
%
%       Here is my information.
%
% Otherwise the top of the section header will not line up with the top
% of the section. Of course, using a single comment character (%) on
% empty lines allows for the function of the first example with the
% readability of the second example.
\renewcommand{\section}[2]%
        {\pagebreak[2]\vspace{1.3\baselineskip}%
         \phantomsection\addcontentsline{toc}{section}{#1}%
         \hspace{0in}%
         \marginpar{
         \raggedright \scshape #1}#2}

% An itemize-style list with lots of space between items
\newenvironment{outerlist}[1][\enskip\textbullet]%
        {\begin{enumerate}[#1]}{\end{enumerate}%
         \vspace{-.6\baselineskip}}

% An itemize-style list with little space between items
\newenvironment{innerlist}[1][\enskip\textbullet]%
        {\begin{compactenum}[#1]}{\end{compactenum}}

% To add some paragraph space between lines.
% This also tells LaTeX to preferably break a page on one of these gaps
% if there is a needed pagebreak nearby.
\newcommand{\blankline}{\quad\pagebreak[2]}

% Make lists without bullets
\renewenvironment{itemize}{
  \begin{list}{}{
      \setlength{\leftmargin}{1em}
    }
}{
  \end{list}
}

%%%%%%%%%%%%%%%%%%%%%%%% End Helper Commands %%%%%%%%%%%%%%%%%%%%%%%%%%%

%%%%%%%%%%%%%%%%%%%%%%%%% Begin CV Document %%%%%%%%%%%%%%%%%%%%%%%%%%%%

\begin{document}
\makeheading{\name}

\section{Contact}
%
% NOTE: Mind where the & separators and \\ breaks are in the following
%       table.
%
% ALSO: \rcollength is the width of the right column of the table 
%       (adjust it to your liking; default is 1.85in).
%
\newlength{\rcollength}\setlength{\rcollength}{1.85in}%
%
\begin{tabular}[t]{@{}p{\textwidth-\rcollength}p{\rcollength}}
\href{http://www.cs.virginia.edu/}%
     {Department of Computer Science} & \\
\href{http://www.seas.virginia.edu/}{School of Engineering} \\
\href{http://www.virginia.edu/}{University of Virginia}
                           & \textit{Phone:} (214) 709-6785 \\
151 Engineer's Way            & \textit{Fax:} (434) 982-2214 \\
P.O. Box 400740           & 
\href{mailto:sookoor@cs.virginia.edu}{sookoor@cs.virginia.edu}\\
Charlottesville, VA  22903-4740 USA  & 
\href{http://www.tamimsookoor.com}{http://www.tamimsookoor.com}\\
\end{tabular}

\section{Security Clearance} 
%
Secret (expires: 2021) 

\section{Citizenship}
%
United States of America

\section{Research Interests}
% 
My research interests cover embedded and networked sensor systems, wireless
sensor networks, Cyber-Physical Systems (CPSs), and wireless embedded
networks. My research has focused on reducing the effort required to build
systems that use sensing, intelligence, and control to bridge the gap between
the cyber- and physical-worlds through the development of compilers, programming
languages, and debuggers that address the special needs posed by such systems. A
second research thrust, that began a few years ago, is an application of CPSs to
improve the energy efficiency of Smart Buildings. 

\section{Education}
%
\href{http://www.virginia.edu/}{\textbf{University of Virginia}}, 
Charlottesville, Virginia USA \\
Adviser: \href{http://www.cs.virginia.edu/~whitehouse/}{Prof. Kamin Whitehouse}
\\
Area of Study: Wireless Embedded Networks
\begin{outerlist}
\item[] PhD, \href{http://www.cs.virginia.edu/}{Computer Science}, May 2012
  (Expected)
 \begin{innerlist}
 \item Dissertation Topic: Retrofitting Centralized HVAC Systems for
   Occupant-Oriented Room-Level Heating and Cooling
 \end{innerlist}
\item[] M.S., \href{http://www.cs.virginia.edu/}{Computer Science},
  December 2009
  \begin{innerlist}
  \item Thesis: The Design of MDB: a Macrodebugger for Wireless
    Embedded Networks
  \end{innerlist}
\end{outerlist}

\blankline

\href{http://www.vanderbilt.edu}{\textbf{Vanderbilt University}},
Nashville, Tennessee USA
\begin{outerlist}
\item[] B.E., 
        \href{http://eecs.vuse.vanderbilt.edu/}
             {Computer Engineering}, May 2006
        \begin{innerlist}
        \item \emph{Summa cum Laude}, Honors in Engineering
        %\item Electrical specialization (emphasis on electromagnetics and digital computers)
        \item Minor in \href{http://www.math.vanderbilt.edu/}
                            {Mathematics} 
        %      (programming and algorithms track)
        \end{innerlist}
\end{outerlist}

\section{Research Experience}
\href{http://www.virginia.edu}{\textbf{University of Virginia}}
\textit{Graduate Research Assistant}%
        %\hfill \textbf{August 2006 to present}

\begin{outerlist}
\item[] Advisor:
      \href{http://www.cs.virginia.edu/~whitehouse}
           {Prof. Kamin Whitehouse}
      \hfill \textbf{August 2006--Present}
        \begin{innerlist}
        \item Implemented an occupancy-based HVAC control system to re-direct
          airflow in homes based on occupant needs in order to improve the
          efficiency of their heating and air conditioning systems.
        \item Developed the Smart Thermostat which learns the activity patterns
          in homes and strategically heats and cools them only when needed.
        \item Implemented
          \href{http://www.cs.virginia.edu/~whitehouse/research/mdb/}{MDB},
          the first macrodebugger for wireless embedded networks.
        \item Implemented \href{http://www.cs.virginia.edu/~whitehouse/research/macrolab/}{MacroLab}, a macroprogramming framework for
          cyber-physical systems.
        \item Helped deploy
          \href{http://www.cs.virginia.edu/~whitehouse/research/testbed/}{VineLab},
          the Olsson Hall wireless testbed at the University of Virginia.
        \item Developed Reliance, a reliable, latency-bound routing
          protocol and link quality
          maintenance technique for cyber-physical systems.
        \item Implemented an activity recognition system using a hip-mounted
          three-axis accelerometer and
          \href{http://www.cs.virginia.edu/~whitehouse/research/sieve/}{Sieve}, an
          event classification framework developed at the University of Virginia.
        \end{innerlist}

\item[] Advisor: \href{http://www.cs.virginia.edu/brochure/profs/stankovic.html}
           {Prof. John Stankovic}
      \hfill \textbf{August 2006--August 2007}
        \begin{innerlist}
        \item Helped develop, implement, and deploy \href{http://www.cs.virginia.edu/wsn/luster/}{LUSTER}, a wireless
          sensor network for environmental research. 
        \end{innerlist}
\end{outerlist}

\blankline

\href{http://www.wsnlabberkeley.com}{\textbf{The Wireless Sensor
    Networks Lab}}
\textit{Research Intern}%

\begin{outerlist}
\item[] Supervisor: Dr. Marco Sgroi
  \hfill \textbf{May--August 2008}
  \begin{innerlist}
    \item Implemented a gesture recognition-based computer interface
      using accelerometers on MicaZ motes and the
      \href{http://spine.tilab.com}{Signal Processing In Node
        Environment (SPINE)} framework.
    \item Collected and analyzed accelerometer and gyroscope data to identify
      projects that could be implemented using SPINE.
    \item Implemented a Matlab-based real-time visualization
      environment for sensor readings using SPINE.
  \end{innerlist}
\end{outerlist}

\blankline

\href{http://www.vanderbilt.edu}{\textbf{Vanderbilt University}}
\textit{Undergraduate Researcher}% 
        %\hfill \textbf{May 2005 to May 2006}

\begin{outerlist}
\item[] Advisor: \href{http://www.vuse.vanderbilt.edu/~koutsoxd/}
                     {Prof. Xenofon Koutsoukos} 
               \hfill \textbf{May 2005--May 2006}
               \begin{innerlist}
               \item Participated in the
                 \href{http://vusrp.vanderbilt.edu/}{Vanderbilt
                   Undergraduate Summer Research Program} in 2005 and continued research as an
                 independent study project the following year.
               \item Implemented a parking space finder
                 service.
               \end{innerlist}
\end{outerlist}

\section{Teaching Experience}
\href{http://www.virginia.edu}{\textbf{University of Virginia}}
\textit{Graduate Teaching Assistant}%
\begin{outerlist}
\item Computer Networks
      \hfill \textbf{Spring 2007}
\item Software Development Methods
      \hfill \textbf{Fall 2006}
\end{outerlist}

\blankline

\href{http://www.vanderbilt.edu}{\textbf{Vanderbilt University}} 
\textit{Tutor}% 
        %\hfill \textbf{May 2005 to May 2006}
\begin{outerlist}
\item Mathematics
      \hfill \textbf{2005--2006}
\end{outerlist}

\section{Scholarships and Fellowships}
%
\href{http://www.virginia.edu}{\textbf{University of Virginia}}
\begin{innerlist}
\item \href{http://sensys.acm.org/2009/}{SenSys'09 Student Travel Award}, 2009
\item DuPont Fellowship, 2008
\item \href{http://www.cse.wustl.edu/~lu/ipsn07.html}{IPSN Student Travel Award}, 2007
\item Graduate Research Assistantship, 2007--Present
\end{innerlist}

\blankline

\href{http://www.vanderbilt.edu}{\textbf{Vanderbilt University}}
\begin{innerlist}
\item \href{http://vusrp.vanderbilt.edu/}{Vanderbilt Undergraduate
  Summer Research Program} Fellowship, 2005
\item
  \href{http://frontweb.vuse.vanderbilt.edu/vuse_web/summerResearch/index.html}{Vanderbilt
  School of Engineering Summer Research Program}, 2005 (declined)
\item \href{http://www.tntech.edu/ece/reu/} {Tennessee Tech REU in Network and
  Communication Systems}, 2005 (declined)
\end{innerlist}

\section{Honors and Awards} 
%
\href{http://www.virginia.edu}{\textbf{University of Virginia}}
\begin{innerlist}
\item \href{http://smart.asee.org/}{ASEE SMART Scholarship Program} Fellow, 2010
\item U.Va. representative to \href{http://www.vacgs.net/forum.html}{VCGS 5th
  Annual Graduate Student Research Forum}, 2010
\end{innerlist}

\blankline

\href{http://www.vanderbilt.edu}{\textbf{Vanderbilt University}}
\begin{innerlist}
\item Dean's Award for Outstanding Scholarship, 2006
\item Dean's List with High Honors, All semesters
\item \href{http://www.tbp.org}{Tau Beta Pi}, National Engineering
  Honor Society,
        2005--Present
\item \href{http://www.hkn.org}{Eta Kappa Nu}, Electrical and Computer
  Engineering Honor Society, 
        2005--Present
\item \href{http://www.mortarboard.org}{Mortar Board}, National
  College Senior Honor Society, 2005--Present
\end{innerlist}

\section{Professional Organization Memberships}
%
\href{http://www.acm.org/}{Association of Computing Machinery} (ACM) \\
\begin{innerlist}
\item \href{http://www.sigsoft.org/}{ACM SIGSOFT}
\end{innerlist}

\blankline

\href{http://www.ieee.org/portal/site}{Institute of Electrical and
  Electronic Engineers}(IEEE) \\
\begin{innerlist}
\item \href{http://www.computer.org/portal/site/ieeecs/index.jsp}{IEEE Computer
  Society} 
\item \href{http://www.comsoc.org/}{IEEE Communications Society}
\end{innerlist}

\section{Publications}
%
\textbf{Tamim Sookoor} and Kamin Whitehouse, ``SmartZone: Room-level HVAC Control using
Residence Occupancy Models,'' (under preparation)

\blankline

\textbf{Tamim Sookoor}, Brian Holben, Kamin Whitehouse. ''Feasibility of
Retrofitting Centralized HVAC Systems for Room-Level Zoning,'' (under submission) 

\blankline

Virginia Smith, \textbf{Tamim Sookoor}, Kamin Whitehouse. ``Modeling Building
Thermal Response to HVAC Zoning,'' to appear in
\href{http://conet2012.cister-isep.info/}{\textit{The Third International
    Workshop on Networks of Cooperating Objects}}, Beijing, China, April 2012.

\blankline

Kamin Whitehouse, Juhi Ranjan, Jiakang Lu, \textbf{Tamim Sookoor}, Carrie Burke,
Galen Staengle, Anselmo Canfora, Hossein Haj-Hariri. ``Towards Occupancy-driven
Heating and Cooling,'' to appear in
\href{http://www.computer.org/portal/web/computingnow/dtcfp4}{\textit{IEEE
    Design \& Test Special Issue on Green Buildings}}, Jul/Aug. 2012.

\blankline

Timothy Hnat, Vijay Srinivasan, Jiakang Lu, \textbf{Tamim Sookoor}, Raymond
Dawson, John Stankovic, Kamin Whitehouse. ``The Hitchhiker's Guide to Successful
Residential Sensing Deployments,'' in
\href{http://sensys.acm.org/2011/}{\textit{The 9th ACM Conference on Embedded
    Networked Sensing Systems (SenSys'11)}}, Seattle, WA.,
Nov. 2011. (acceptance ratio 19.5\%)

\blankline

Jiakang Lu, \textbf{Tamim Sookoor}, Gao Ge, Vijay Srinivasan, Brian Holben, John
Stankovic, Eric Field, and Kamin Whitehouse, ``The Smart Thermostat: Using
Occupancy Sensors to Save Energy in Homes,'' in
\href{http://sensys.acm.org/2010/}{\textit{The 8th ACM Conference on Embedded
    Networked Sensor Systems (SenSys 2010)}}, Zurich, Switzerland,
Nov. 2010. (acceptance ratio 17\%)

\blankline

Timothy Hnat, \textbf{Tamim Sookoor}, Pieter Hooimeijier, Westley Weimer, and
Kamin Whitehouse, ``A Modular and Extensible Macroprogramming Compiler,'' in
\href{http://www.sesena.info/}{\textit{Workshop on Software Engineering for
    Sensor Network Applications (SESENA 2010)}}, Cape Town, South Africa, May
2010.

\blankline

\textbf{Tamim Sookoor}, Timothy Hnat, Pieter Hooimeijer, Westley Weimer, and Kamin
Whitehouse, ``Macrodebugging: Global Views of Distributed Program Execution,''
in \href{http://sensys.acm.org/2009/}{\textit{The 7th ACM Conference on Embedded
Networked Sensor Systems (Sensys 2009)}}, Berkeley, CA, Nov. 2009. (acceptance
ratio 17.6\%)

\blankline

Timothy W. Hnat, \textbf{Tamim I. Sookoor}, Pieter Hooimeijer, Westley
Weimer,  and Kamin Whitehouse, ``MacroLab: A Vector-based
Macroprogramming Framework for Cyber-Physical Systems,'' in
\href{http://sensys.acm.org/2008/}{\textit{The 6th ACM Conference on Embedded
    Networked Sensor Systems (SenSys 2008)}}, Raleigh, NC,
Nov. 2008. (acceptance ratio 16\%)

\blankline

L.~Selavo, A.~Wood, Q.~Cao, \textbf{T.~Sookoor}, H.~Liu, A.~Srinivasan, Y.~Wu, W.~Kang,
J.~Stankovic, D.~Young, J.~Porter, ``LUSTER: Wireless Sensor Network for
Environmental Research,'' in
\href{http://sensys.acm.org/2007/Home.html}{\textit{The 5th ACM Conference on
Embedded Networked Sensor Systems (SenSys 2007)}}, Sydney, Australia,
Nov. 2007. (acceptance ratio 16.8\%)

\section{Demos}
%
Timothy W. Hnat, \textbf{Tamim I. Sookoor}, and Kamin Whitehouse ``Demo Abstract:
Macrodebugging with MDB,'' in \href{http://sensys.acm.org/2009/}{\textit{The 7th
    ACM Conference on Embedded Networked Sensor Systems (SenSys 2009)}},
Berkeley, CA, Nov. 2009. \\

\textbf{Tamim I. Sookoor}, Timothy W. Hnat, and Kamin Whitehouse, ``Demo
Abstract: Programming Cyber-Physical Systems with MacroLab,'' in
\href{http://sensys.acm.org/2008/}{\textit{The 6th ACM Conference on
    Embedded Networked Sensor Systems (SenSys 2008)}}, Raleigh, NC, Nov. 2008.

\section{Grant Writing Activity}
%
\textbf{Office of Emerging Frontiers in Research and Innovation (EFRI)},
\textit{EFRI-SEED: Occupant Oriented Heating and Cooling}
\\
PI: Cameron (Kamin) Whitehouse. My role: Prepared figures and graphs and
assisted with proofreading and editing the proposal. \\
Total Amount: \$1,999,642. Awarded: September 1, 2010. 

\blankline


\textbf{University of Virginia Double 'Hoo Research Grant}, \textit{Title:
Reducing the National Energy Consumption Through Occupancy-based HVAC Control}
\\
PI: Tamim Sookoor. \\
Total Amount: \$5,000. Submitted: February 19, 2010. 

\blankline

\textbf{NSF Computer and Network Systems (CNS)}, \textit{NeTS: Small: Time Travel Debugging for Wireless Embedded Networks}
\\
PI: Cameron (Kamin) Whitehouse. My role: Prepared first draft of certain
subsections of research approach and added references; and edited drafts of
cumulative proposal. \\
Submitted: December 17, 2009.

\section{Conferences Attended}
\href{http://www.computational-sustainability.org/compsust10}{2nd International
  Conference on Computational Sustainability} (CompSust'10) \\
\href{http://sensys.acm.org/2009/}{The 7th ACM Conference on Embedded Networked
Sensor Systems} (SenSys 2009) \\
\href{http://sensys.acm.org/2008/}{The 6th ACM Conference on Embedded Networked
  Sensor Systems} (SenSys 2008)\\
\href{http://www.cse.wustl.edu/~lu/ipsn07.html}{International Conference on
  Information Processing in Sensor Networks} (IPSN 2007)

\section{Service}
%
\textbf{Conference Reviewing}
\begin{innerlist}
\item \href{http://varma.ece.cmu.edu/ICCPS/}{First International Conference on
Cyber-Physical Systems (ICCPS 2010)} 
\item \href{http://www.inss-conf.org/2009/}{
Sixth International Conference on Networked Sensing Systems (INSS 2009)} 
\item \href{http://www.ieee-secon.org/2009/}{6th Conference on Sensor, Mesh and Ad Hoc Comm. and Networks (SECON 2009)} 
\item \href{http://www.the-internet-of-things.org/}{Internet of Things
Conference 2008} 
\end{innerlist}

\blankline

\textbf{Demonstrations}
\begin{innerlist}
\item \textbf{Computer Science Day / SEAS Open House 2008, 2009, 2010}:
  Presented demos and posters to encourage high school students to engage in
  computer science. 
\item \textbf{Computer Technologies Career Academy 2009}: Presented hands-on
  demonstrations of wireless sensor network applications to middle school students.
\end{innerlist}

\section{Technical Skills} 
%
\textbf{Programming Languages:} C, C++, nesC, Java, HTML, Python, Perl, PHP,
Lisp, UNIX shell scripting, SQL, SVN, Assembly

\blankline

\textbf{Application Programs:} Matlab, Mathematica, MS Visual Studio 6.0/.Net, Eclipse

\blankline

\textbf{Operating Systems:} TinyOS, $\mu$C/OS, Linux, Microsoft Windows, OS X 

\blankline

\textbf{Hardware:} Xbow and Sentilla motes, Tektronix and National
Instruments hardware

\section{Coursework} 
%
\textbf{Graduate:} Computer Organization, Theory of Computation, Algorithms,
From Sensors to Scientists: Applications of Sensor Networks, Sensor Networks,
Applied Statistics for Engineers and Scientists, Programming Paradigms for
Wireless Embedded Systems, Operating Systems, Special Topics in Wireless Sensor
Networks, Applications of CPSs

\blankline

\textbf{Undergraduate:} Programming Languages, Program Design and Data
Structures, Principles of Operating Systems I, Computer Organization,
Probability and Statistics for Engineers, Microcontrollers, Embedded
Systems, Signals and Systems, Discrete Structures, Intermediate
Software Design

\section{References}
\begin{minipage}[t]{0.5\textwidth}
\begin{itemize}
\item Kamin Whitehouse \\ Assistant Professor \\ University of Virginia \\
  (434) 982-2211 \\
  \href{mailto:whitehouse@cs.virginia.edu}{\tt whitehouse@cs.virginia.edu}
\end{itemize}
\end{minipage}
\begin{minipage}[t]{0.5\textwidth}
\begin{itemize}
\item John Stankovic \\ BP America Professor \\ University of Virginia \\
  (434) 982-2275 \\
  \href{mailto:stankovic@cs.virginia.edu}{\tt stankovic@cs.virginia.edu}
\end{itemize}
\end{minipage}

\bigskip

%Footer
\begin{center}
\begin{footnotesize}
Last updated: \today \\
%\href{http://www.cs.virginia.edu/sookoor/docs/tamimSookoorCV.pdf}{\tt
%  http://www.cs.virginia.edu/sookoor/docs/tamimSookoorCV.pdf}
\end{footnotesize}
\end{center}
\end{document}

%%%%%%%%%%%%%%%%%%%%%%%%%% End CV Document %%%%%%%%%%%%%%%%%%%%%%%%%%%%%
